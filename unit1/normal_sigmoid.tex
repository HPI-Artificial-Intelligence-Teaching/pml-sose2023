\documentclass[a4paper]{article}
\usepackage[square,sort,comma,numbers]{natbib}
\usepackage{blindtext} % Package to generate dummy text
\usepackage{charter} % Use the Charter font
\usepackage[utf8]{inputenc} % Use UTF-8 encoding
\usepackage[T1]{fontenc}
\usepackage{microtype} % Slightly tweak font spacing for aesthetics
\usepackage{amsthm, amsmath, amssymb} % Mathematical typesetting
\usepackage{float} % Improved interface for floating objects
\usepackage{hyperref} % For hyperlinks in the PDF
\usepackage{graphicx, multicol} % Enhanced support for graphics
\usepackage{xcolor} % Driver-independent color extensions
\usepackage{pseudocode} % Environment for specifying algorithms in a natural way
\usepackage{datetime} % Uses YEAR-MONTH-DAY format for dates

\addtolength{\hoffset}{-2.25cm}
\addtolength{\textwidth}{4.5cm}
\addtolength{\voffset}{-3.25cm}
\addtolength{\textheight}{5cm}

\setlength{\parskip}{1ex}
\setlength{\parindent}{0in}

\DeclareUnicodeCharacter{03B1}{$\alpha$}

\usepackage{fancyhdr} % Headers and footers
\pagestyle{fancy} % All pages have headers and footers
\fancyhead{}\renewcommand{\headrulewidth}{0pt} % Blank out the default header
\fancyfoot[L]{} % Custom footer text
\fancyfoot[C]{} % Custom footer text
\fancyfoot[R]{\thepage} % Custom footer text
\newcommand{\note}[1]{\marginpar{\scriptsize \textcolor{red}{#1}}} % Enables comments in red on margin

%----------------------------------------------------------------------------------------

\begin{document}

%-------------------------------
%	TITLE SECTION (do not modify unless you really need to)
%-------------------------------
\fancyhead[C]{}
\hrule \medskip
\begin{minipage}{0.295\textwidth}
    \raggedright
    \hfill\\
    % \footnotesize
    % Start: 24.4.2023\\
    % Return: 19.5.2023 \hfill\\
    % \href{https://classroom.github.com/classrooms/124387260-hpi-artificial-intelligence-teaching-pml-classroom}{https://classroom.github.com}
\end{minipage}
\begin{minipage}{0.4\textwidth}
    \centering
    \large
    Note on Lecture 2\\
    \normalsize
    Matching the Probit and Logit Function\\
\end{minipage}
\begin{minipage}{0.295\textwidth}
    \raggedleft
    \hfill\\
\end{minipage}
\medskip\hrule
\bigskip

%-------------------------------
%	ASSIGNMENT CONTENT 
%-------------------------------

\section*{Background}
In the lecture, we have seen both the logit and Probit functions defined by
\begin{align}
    \mathrm{logit}(x) & := \frac{\exp(x)}{1+\exp(x)} \,, \label{eq:logit} \\
    \Phi(x) & := \int_{-\infty}^x \frac{1}{\sqrt{2\pi}} \exp\left(-\frac{t^2}{2}\right)\, dt \label{eq:probit} \,. 
\end{align}
Both of these functions have the property that they maps the real line $\mathbb{R}$ to the interval $(0,1)$ with a monotonically increasing function. What we would like to show is that the two function have the same derivative at zero if we use $\mathrm{logit}(x)$ and $\Phi\left(\sqrt{\frac{\pi}{8}}x\right)$.

\section*{Derivation}
In order to derive the above equation, let us first derive the first derivative of the logit function \eqref{eq:logit}. 
\begin{align*}
    \frac{d\ \mathrm{logit}(x)}{dx} & = \frac{d\exp(x)\cdot(1+\exp(x))^{-1}}{dx} \\
    & = \frac{d\exp(x)}{dx}\cdot(1+\exp(x))^{-1} + \exp(x)\cdot \frac{d(1+\exp(x))^{-1}}{dx} \\
    & = \exp(x)\cdot(1+\exp(x))^{-1} + \exp(x)\cdot \left(-(1+\exp(x))^{-2}\exp(x)\right) \\
    & = \exp(x)\cdot(1+\exp(x))^{-1}\cdot\left[1 - \exp(x)\cdot(1+\exp(x))^{-1}\right] \\
    & = \mathrm{logit}(x)\cdot\left[1 - \mathrm{logit}(x)\right] \, 
\end{align*}
where the second line follows from the product rule of differentiation, the third line from carrying out the individual derivatives, the fourth line from factoring out the coming factor $\exp(x)\cdot(1+\exp(x))^{-1}$ and the final line from reusing the definition \eqref{eq:logit}.

As a next step, let us now derive the first derivative of the Probit function \eqref{eq:probit} when we scale the input argument by a factor $c$.
\begin{align*}
    \frac{d\Phi(c\cdot x)}{dx} & =  \frac{d\Phi(c\cdot x)}{d(c\cdot x)} \cdot \frac{d(c\cdot x)}{dx} \\
    & = \frac{1}{\sqrt{2\pi}} \exp\left(-\frac{(c\cdot x)^2}{2}\right) \cdot c \,,
\end{align*}
where the first line follows from the chain rule of differentiation and the second line from carrying out the individual derivatives.

Thus, in order to match the derivative at zero, we need to find the value $c$ such as that 
\begin{align*}
    \left.\frac{d\Phi(c\cdot x)}{dx}\right|_{x=0} & = \left.\frac{d\ \mathrm{logit}(x)}{dx}\right|_{x=0} \\
    \frac{1}{\sqrt{2\pi}} \exp\left(-\frac{(0\cdot c)^2}{2}\right) \cdot c & = \mathrm{logit}(0)\cdot\left[1 - \mathrm{logit}(0)\right] \\
    \frac{1}{\sqrt{2\pi}} \cdot c & = \frac{1}{4} \\
    c & = \sqrt{\frac{2\cdot \pi}{16}} = \sqrt{\frac{\pi}{8}} \,,
\end{align*}
which proves that $\mathrm{logit}(x)$ and $\Phi\left(\sqrt{\frac{\pi}{8}}x\right)$ have the same derivative at zero.

\end{document}
